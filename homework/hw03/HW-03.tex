\documentclass[12pt]{elsart}
\usepackage{amsmath}
\usepackage{amssymb}
\usepackage{program}
\newcommand{\field}[1]{\mathbb{#1}}

\usepackage{algorithm}
\usepackage{algpseudocode}

%%%%%%%%%%%%%%%%%%%%%%%%%%%%%%%%%%%%%%%%%Space to make more readable!
%\vspace{10 mm}
%%%%%%%%%%%%%%%%%%%%%%%%%%%%%%%%%%%%%%%%%Take out later!

\begin{document}

\pagestyle{empty}

\begin{center}
\Large  CS3343 Analysis of Algorithms Fall 2017 \\
\large {\bf Homework 3}\\
\normalsize Due 9/24/17 before 11:59pm (Central Time)
\end{center}

{\bf 1.  Probability (7 points)}

For the following problem, clearly describe the sample space and the random variables you use.  Be sure to justify where you get your expected values from.

Consider playing a game where you roll $n$ fair six-sided dice.  For every 1 or 6 you roll you win \$30, for rolling any other number you lose \$$3 n$.

\begin{enumerate}
   \item  First assume $n=1$ (i.e., you only roll one six-sided die). 
\begin{enumerate}
   	\item  (1 point) Describe the sample space for this experiment.\\
       The sample space consists of the set of possible die rolls, i.e. $\{1, 2, 3, 4, 5, 6\}$
  	 \item (1 point) Describe a random variable which maps an outcome of this experiment to the winnings you receive.\\
       Random variable $X$.\\
       \begin{tabular}{|l|l|l|l|l|l|l|}
           \hline
           s & 1 & 2 & 3 & 4 & 5 & 6\\
           \hline
           X(s) & 30 & -3 & -3 & -3 & -3 & 30\\
           \hline
       \end{tabular}
  	 \item (1 point)  Compute the expected value of this random variable.
       \begin{flalign*}
           E[X] &= \sum_{s=1}^6 X(s) \cdot \frac{1}{6}&\\
                &= \frac{1}{6} (30 + (-3) + (-3) + (-3) + (-3) + 30)\\
                &= \frac{48}{6}\\
                &= 8
       \end{flalign*}
\end{enumerate}
   \item  Now assume $n=6$ (i.e., you roll six six-sided dice). 
\begin{enumerate}
   	\item (1 point) Describe the sample space for this experiment (you don't need to list the elements but describe what is contained in it).\\
       The sample space consists of the set of all permutations of numbers rolled from six separate 6-sided dice.
   	\item (1 point) Describe a random variable which maps an outcome of this experiment to the winnings you receive. (Hint: Express your random variable as the sum of six random variables.)\\
       Let $X$ be the random variable that maps the outcome to the winnings for six dice and $X_i$ be the random variable that maps the outcome to the winnings for a single die.\\
       $X = \sum\limits_{i=1}^6 X_i$
       
\newpage

 	  \item  (1 point) Compute the expected value of this random variable using the linearity of expectation.  Based on this would you play this game?\\
        \begin{tabular}{|l|l|l|l|l|l|l|}
            \hline
            s & 1 & 2 & 3 & 4 & 5 & 6\\
            \hline
            $X_i(s)$ & 30 & -18 & -18 & -18 & -18 & 30\\
            \hline
        \end{tabular}\\
        \begin{flalign*}
            E[X_i] &= \frac{2}{6} \cdot 30 + \frac{4}{6} \cdot (-18)&\\
                 &= \frac{60}{6} - \frac{72}{6}\\
                 &= 10 - 12 = -2
        \end{flalign*}
        \begin{flalign*}
            E[X] &= E\left[\sum\limits_{i=1}^6 X_i\right]&\\
                 &= \sum\limits_{i=1}^6 E[X_i]\\
                 &= 6E[X_i]\\
                 &= 6(-2) = -12
        \end{flalign*}
\end{enumerate}
   \item  (1 point) What is the largest value of $n$ for which you would still want to play the game?  Justify your answer.\\
   Would want $E[X] > 0$ otherwise you are expected to lose money.\\
   Solving for $n$ and using the above inequality:
   \begin{flalign*}
       E[X] &> 0&\\
       \frac{2}{6} \cdot 30 + \frac{4}{6} \cdot (-3n) &> 0\\
       10 - 2n &> 0\\
       \frac{10}{2} &> n\\
       5 &> n
   \end{flalign*}
   Would play the game for $n < 5$ so the largest $n$ where I would still have an advantage is $n = 4$.
\end{enumerate}

{\bf 2. Variants of quicksort (6 points)}

Suppose company X has created 3 new variants of quicksort but, because they are unsure which is asymptotically best, they have hired you to analyze their runtimes.  Compute the asymptotic runtimes of the variants and use this to justify which is best:

\begin{itemize}
 \item  Variant 1: Partitioning the array still takes $\Theta (n)$ time but the array will always be divided into (at least) a 10\% and  90\% portion.\\
    $T(n) = T(n/10) + T(9n/10) + \Theta(n)$\\
    Drawing a recursion tree will show that it is heavy towards the right and $cn$ work is done at each level.

    The min height on the left is computed by:
    \begin{flalign*}
        \frac{n}{10^h} &= 1&\\
                     n &= 10^h\\
           \log_{10} n &= h
    \end{flalign*}
    The max height on the right is computed by:
    \begin{flalign*}
        \frac{n}{(\frac{10}{9})^h} &= 1&\\
                                 n &= \left(\frac{10}{9}\right)^h\\
             \log_{\frac{10}{9}} n &= h
    \end{flalign*}
    So $cn\log_{10} n \leq T(n) \leq cn\log_{\frac{10}{9}} n \in \Theta(n\log n)$
 \item  Variant 2: Partitioning the array now takes $\Theta (n^{1.1})$ time but the array will always be divided perfectly in half.\\
    $T(n) = 2T(n/2) + \Theta(n^{1.1})$\\
    Master Theorem Case 3:\\
    $n^{\log_2 2} = n$\\
    $n^{1.1} \in \Omega(n^{1 + \epsilon})$ for $\epsilon = 0.1 > 0$
    $n^{1.1} \in \Omega(n^{1.1})$ is trivially true.\\
    Regularity condition:\\
    \begin{flalign*}
        2\left(\frac{n}{2}\right)^{1.1} &\leq Cn^{1.1}&\\
              \frac{1}{2^{0.1}} n^{1.1} &\leq Cn^{1.1}
    \end{flalign*}
    Choose $C = \frac{1}{2^{0.1}} < 1$, so $T(n) \in \Theta(n^{1.1})$
 \item  Variant 3: Partitioning the array now only takes $\Theta (\sqrt{n})$ time but all of the numbers except the pivot will be partitioned into a single array.\\
    $T(n) = T(n - 1) + \Theta(\sqrt{n})$\\
    With a recursion tree, one can see the runtime is:\\
    $\sqrt{n} + \sqrt{n - 1} + \ldots + \sqrt{2} + \sqrt{1} = \sum\limits_{i=1}^{n} \sqrt{i} \in \Theta(n^{3/2})$ so $T(n) \in \Theta(n^{3/2})$
\end{itemize}

Since $\log n \in o(n^c)$ for all $c > 0$:\\
$\Theta(n\log n) < \Theta(n^{3/2}) = \Theta(n \cdot n^{0.5})$\\
$\Theta(n\log n) < \Theta(n^{1.1}) = \Theta(n \cdot n^{0.1})$\\
So the one with the best runtime is Variant 1 which is $\Theta(n\log n)$

{\bf Hint:} It may help to know that $\sum\limits_{i=1}^n \sqrt{n} \in \Theta(n^{3/2})$ and that $\log n \in o(n^c)$ for all $c>0$.

{\bf 3. Linear search (6 points)}

The code below searches an unsorted array, $A$, for a given value.  It returns the index of the value if it is present.  Otherwise it returns $-1$.

\begin{algorithm}
\caption{int linearSearch(int $A[0\ldots n-1]$, int $val$)}
 \begin{algorithmic}
 \State $i = 0$;
 \While{$i< n$}
    \If{$val==A[i]$}
       \State return $i$;
    \EndIf
    \State $i++$;
  \EndWhile
 \State return $-1$;
\end{algorithmic}
\end{algorithm}

Suppose before we call linearSearch we randomly permute (i.e., reorder) the elements in $A$.  In the following problem, we will find the expected asymptotic runtime of linearSearch.

\begin{enumerate}
 \item (1 point) What is the best case asymptotic runtime of linearSearch on an array of $n$ elements?  What index must $val$ be located at to elicit this runtime?\\
    The best case runtime is $\Theta(1)$ where $val$ is located at index 1 and so returns immediately.
 \item (1 point) What is the worst case asymptotic runtime of linearSearch on an array of $n$ elements?  What index must $val$ be located at to elicit this runtime?\\
    The worst case runtime is $\Theta(n)$ where $val$ is located at index $n - 1$ or it is not in the array at all and so must examine the whole array.
 \item (1 point) For the sake of simplicity, assume $val$ is in $A$.  Let $X_j$ be the random variable which, given a permutation of the elements in our array, is $1$ if index $A[j]$ will be checked in our linearSearch for that permutation and $0$ otherwise.  What is the expected value of $X_j$?\\
    $E[X_j] = 1 \cdot \frac{1}{n} + 0 \cdot \frac{n - 1}{n} = \frac{1}{n}$
 \item (1 point) Let $X$ be the random variable which, given a permutation of the elements in our array,  is the number of elements of $A$ which will be checked in our linearSearch for that permutation.  Define $X$ using $n$ copies of the random variable defined in the previous step.\\
    $X = \sum\limits_{j=1}^n j \cdot X_j$
    
\newpage
    
 \item (2 points) Compute the expected value of $X$ for using the linearity of expectation. Given this, what is the expected asymptotic runtime of linearSearch?
    \begin{flalign*}
        E[X] &= E\left[\sum\limits_{j=1}^n j \cdot X_j\right]&\\
             &= \sum\limits_{j=1}^n j \cdot E[X_j]\\
             &= \frac{n(n + 1)}{2} \cdot \frac{1}{n}\\
             &= \frac{n + 1}{2}             
    \end{flalign*}
    
    $E[X] = \frac{n + 1}{2} \in \Theta(n)$ so the expected runtime is $\Theta(n)$
\end{enumerate}

\end{document}