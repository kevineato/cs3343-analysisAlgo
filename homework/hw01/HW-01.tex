\documentclass[12pt]{elsart}
\usepackage{amsmath}
\usepackage{amssymb}
\usepackage{program}
\newcommand{\field}[1]{\mathbb{#1}}

\usepackage{algorithm}
\usepackage{algpseudocode}

%\vspace{10 mm}

\begin{document}

\pagestyle{empty}

\begin{center}
\Large  CS3343 Analysis of Algorithms Fall 2017 \\
\large {\bf Homework 1}\\
\normalsize Due 9/8/17 before 11:59pm
\end{center}

Justify all of your answers with comments/text in order to receive full credit.  Completing the assignment in Latex will earn you extra credit on Midterm 1.

{\bf 1. Sorting (10 points)}

Consider the sorting algorithm below which
sorts the array $A[1\ldots n]$ into increasing order by repeatedly bubbling the
minimum element of the remaining array to the left.

\begin{algorithm}
\caption{mysterysort(int $A[1\ldots n]$)}
 \begin{algorithmic}
 %
 \State $i = 1$ // C1 --- 1
 \While{$i<=n$} // C2 --- n + 1
 \State //(I) The elements in $A[1 \ldots (i-1)]$  of the array are in sorted order and are all smaller than the elements in $A[i \ldots n]$
    \State $k = n$; // C3 --- n
		\State // $t_{k} =$ \# of times while loop executes
    \While{$(k >= i+1)$} // C4 --- $\sum\limits_{1}^{n} (t_{k} + 1)$
	\State //Inner while loop moves the smallest element in $A[i \ldots n]$ to $A[i]$
          \If{$A[k-1]>A[k]$} // C5 --- $\sum\limits_{1}^{n} t_{k}$
              \State swap $A[k]$ with $A[k-1]$ // C6 --- Worst Case: $\sum\limits_{1}^{n} t_{k}$ Best Case: 0
          \EndIf
       \State $k--$; // C7 --- $\sum\limits_{1}^{n} t_{k}$
     \EndWhile
    \State $i++$; // C8 --- n
  \EndWhile
\end{algorithmic}
\end{algorithm}

\begin{enumerate}
  \item (2 points) Consider running the above sort on the array $[5,3,1,4,2]$.  Show the sequence of changes which the algorithm makes to the array.
	\begin{tabular}{ |l|l|l|l|l| }
		\hline
		5 & 3 & 1 & 4 & 2\\
		\hline
	\end{tabular}
	$\longrightarrow$
	\begin{tabular}{ |l|l|l|l|l| }
		\hline
		1 & 5 & 3 & 2 & 4\\
		\hline
	\end{tabular}
	$\longrightarrow$
	\begin{tabular}{ |l|l|l|l|l| }
		\hline
		1 & 2 & 5 & 3 & 4\\
		\hline
	\end{tabular}
	$\longrightarrow$\\
	\begin{tabular}{ |l|l|l|l|l| }
		\hline
		1 & 2 & 3 & 5 & 4\\
		\hline
	\end{tabular}
	$\longrightarrow$
	\begin{tabular}{ |l|l|l|l|l| }
		\hline
		1 & 2 & 3 & 4 & 5\\
		\hline
	\end{tabular}
  \item (4 points) Use induction to prove the loop invariant (I) is true and then use this to prove the correctness of the algorithm.  Specifically complete the following:
\begin{enumerate}
   \item Base case
		\begin{itemize}
		\item[$\rightarrow$] $i = 1 \Rightarrow$ No elements in $A[1 \ldots i - 1]$, so loop invariant trivially holds true
		\end{itemize}
\newpage
   \item Inductive step\\  ({\bf Hint:} you can assume that the inner loop moves the smallest element in $A[i \ldots n]$ to $A[i]$)
		\begin{itemize}
		\item[$\rightarrow$] $i_{old} =$ value before iteration
		\item[$\rightarrow$] $A[1 \ldots i_{old} - 1]$ are sorted and smaller than $A[i_{old} \ldots n]$
		\item[$\rightarrow$] Inner loop moves the smallest element in $A[i_{old} \ldots n]$ to $A[i_{old}] \Rightarrow A[1 \ldots i_{old}]$ are now in sorted order and smaller than $A[i_{old} + 1 \ldots n]$
		\item[$\rightarrow$] Since $i_{new} = i_{old} + 1$, $A[1 \ldots i_{new} - 1]$ are sorted and smaller than $A[i_{new} \ldots n]$, thus maintaining the loop invariant after iteration
		\end{itemize}
   \item Termination step
		\begin{itemize}
		\item[$\rightarrow$] Terminates when $i > n$, i.e. $i = n + 1$
		\item[$\rightarrow$] Loop invariant says $A[1 \ldots i - 1]$ are sorted and smaller than $A[i \ldots n]$
		\item[$\rightarrow$] $i = n + 1 \Rightarrow A[1 \ldots n]$ are sorted and smaller than $A[n + 1 \ldots n]$, i.e. the entire array is now sorted and the algorithm terminates correctly producing a sorted array
		\end{itemize}
\end{enumerate}
   \item (2 points) Give the best-case and worst-case runtimes of this sort in asymptotic (i.e., $O$) notation.\\
		\textit{Worst Case:}
		\begin{itemize}
		\item[$\rightarrow$] Array is in reverse order $\Rightarrow$ $C1 + (C3 + C8)n + C2(n + 1) + (C5 + C6 + C7)\sum\limits_{1}^{n} t_{k} + C4\sum\limits_{1}^{n} (t_{k} + 1)$
		\item[$\rightarrow$] $C4\sum\limits_{1}^{n} (t_{k} + 1) = C4n + C4\sum\limits_{1}^{n} t_{k}$
		\item[$\rightarrow$] $t_{k} = n - k \Rightarrow \sum\limits_{1}^{n} (n - k) = n - 1 + n - 2 + ... + 2 + 1 \approx \frac{n(n - 1)}{2}\in O(n^2)$
		\end{itemize}
		\textit{Best Case:}
		\begin{itemize}
		\item[$\rightarrow$] Array is in sorted order $\Rightarrow$ $C1 + (C3 + C8)n + C2(n + 1) + (C5 + C7)\sum\limits_{1}^{n} t_{k} + C4\sum\limits_{1}^{n} (t_{k} + 1) + C6\sum\limits_{1}^{n} 0$
		\item[$\rightarrow$] Only C6 is not executed, so the runtime is the same as worst case $\Rightarrow$ $f(n)\in O(n^2)$
		\end{itemize}
\end{enumerate}

{\bf 2. Asymptotic Notation (8 points)}

Show the following using the definitions of $O$, $\Omega$, and $\Theta$.

\begin{enumerate}
   \item (2 points) $2n^3+n^2+4\in \Theta(n^3)$\\
		\textit{$f(n)\in O(n^3)$:}
		\begin{itemize}
		\item[$\rightarrow$] $2n^3 + n^2 + 4 \leq 2n^3 + n^3 + 4n^3 = 7n^3$
		\item[$\rightarrow$] $7n^3 \leq Cn^3$ for $C = 7$ and $n \geq n_{0} = 1 > 0 \Rightarrow f(n)\in O(n^3)$
		\end{itemize}
		\textit{$f(n)\in \Omega(n^3)$:}
		\begin{itemize}
		\item[$\rightarrow$] $2n^3 + n^2 + 4 \geq n^3 + 0 + 0 = n^3$
		\item[$\rightarrow$] $n^3 \geq Cn^3$ for $C = 1$ and $n \geq n_{0} = 1 > 0 \Rightarrow f(n)\in \Omega(n^3)$
		\end{itemize}
		Since $f(n)\in O(n^3)$ and $f(n)\in \Omega(n^3) \Rightarrow f(n)\in \Theta(n^3)$
   \item (2 points) $3n^4-9n^2+4n\in \Theta(n^4)$\\  ({\bf Hint:} careful with the negative number)\\
		\textit{$f(n)\in O(n^4)$:}
		\begin{itemize}
		\item[$\rightarrow$] $3n^4 - 9n^2 + 4n \leq 3n^4 - 0 + 4n(n^3) = 3n^4 + 4n^4 = 7n^4$
		\item[$\rightarrow$] \verb!             !$\leq 7n^4$ for $C = 7$ and $n_0 = 1 \Rightarrow f(n)\in O(n^4)$
		\end{itemize}
		\textit{$f(n)\in \Omega(n^4)$:}
		\begin{itemize}
		\item[$\rightarrow$] $3n^4 - 3 \cdot 3n^2 + 4n \geq 3n^4 - n \cdot n \cdot n^2 + 0 = 3n^4 - n^4 = 2n^4$
		\item[$\rightarrow$] \verb!               !$\geq 2n^4$ for $C = 2$ and $n_0 = 3 \Rightarrow f(n)\in \Omega(n^4)$
		\end{itemize}
		Since $f(n)\in O(n^4)$ and $f(n)\in \Omega(n^4) \Rightarrow f(n)\in \Theta(n^4)$
   \item (4 points) Suppose $f(n)\in O(g_1(n))$ and $f(n)\in O(g_2(n))$.  Which of the following are true?  Justify your answers using the definition of $O$.  Give a counter example if it is false.
\begin{enumerate}
   \item $f(n)\in O( \ 5*g_1(n)+100\ )$
		\begin{itemize}
		\item[$\rightarrow$] $f(n)\in O(g_1(n)) \Rightarrow f(n) \leq C \cdot g_1(n)$ for $C = k_1$ and $n_0 = k_2$
		\item[$\rightarrow$] \verb!                     !$\leq 5 \cdot k_1 \cdot g_1(n) \leq 5 \cdot k_1 \cdot g_1(n) + 100$
		\item[$\rightarrow$] By definition $f(n)\in O(5g_1(n) + 100)$ for $C = k_1$ and $n_0 = k_2$
		\end{itemize}
   \item $f(n)\in O( \ g_1(n)+g_2(n)\ )$
		\begin{itemize}
		\item[$\rightarrow$] $f(n)\in O(g_1(n)) \Rightarrow f(n) \leq C_1 \cdot g_1(n)$ for $C_1 = k_1$ and $n_0 = k_2$
		\item[$\rightarrow$] $f(n)\in O(g_2(n)) \Rightarrow f(n) \leq C_2 \cdot g_2(n)$ for $C_2 = m_1$ and $n_0 = m_2$
		\item[$\rightarrow$] $f(n) \leq k_1 \cdot g_1(n) + m_1 \cdot g_2(n) \leq k_1 \cdot m_1(g_1(n) + g_2(n))$
		\item[$\rightarrow$] By definition $f(n)\in O(g_1(n) + g_2(n))$ for $C = k_1 \cdot m_1$ and $n_0 = max(k_2, m_2)$
		\end{itemize}
   \item $f(n)\in O( \ \frac{g_1(n)}{g_2(n)}\ )$
		\begin{itemize}
		\item[$\rightarrow$] False, Counterexample: $f(n) = n$, $g_1(n) = n$, $g_2(n) = n^2$
		\item[$\rightarrow$] $f(n)\in O(n)$, $f(n)\in O(n^2)$, $f(n)\notin O(\frac{n}{n^2}) \Rightarrow f(n)\notin O(n^{-1})$
		\end{itemize}
   \item $f(n)\in O( \ \max(g_1(n), g_2(n))\ )$
		\begin{itemize}
		\item[$\rightarrow$] Trivially true, since $f(n) \leq C_1 \cdot g_1(n)$ and $f(n) \leq C_2 \cdot g_2(n)$
		\item[$\rightarrow$] Clearly, $f(n)$ will be less than or equal to the larger of the two functions.
		\item[$\rightarrow$] Thus, $f(n)\in O(max(g_1(n), g_2(n)))$
		\end{itemize}
\end{enumerate}
\end{enumerate}

{\bf 3. Summations (4 points)}
Find the order of growth of the following sums.
\begin{enumerate}
\item $\sum\limits_{i=5}^{n} (4i+1). $
	\begin{itemize}
	\item[$\Rightarrow$] $4 \cdot \sum\limits_{i=5}^{n} i + \sum\limits_{i=5}^{n} 1$
	\item[$\Rightarrow$] $4(\sum\limits_{i=1}^{n} i - \sum\limits_{i=1}^{4} i) + (n - 4) = 4(\frac{n(n + 1)}{2} - 10) + (n - 4)$
	\item[$\Rightarrow$] $2n^2 + 3n - 44 \Rightarrow$ sum is $O(n^2)$
	\end{itemize}

\item $\sum\limits_{i=0}^{ \log_2 (n) } 2^i$  (for simplicity you can assume $n$ is a power of 2)
	\begin{itemize}
	\item[$\Rightarrow$] From formula sum $= \frac{1 - 2^{\log_2(n) + 1}}{1 - 2}$
	\item[$\Rightarrow$] $\frac{1 - 2^{\log_2(n)} \cdot 2}{-1} = \frac{1 - 2n}{-1} = 2n - 1 \Rightarrow$ sum is $O(n)$
	\end{itemize}
\end{enumerate}


\end{document}